\documentclass[a4paper]{article}
\usepackage{xgreek}
\usepackage{xltxtra}
\usepackage{setspace}
\usepackage{graphicx}
\newcommand{\HRule}{\rule{\linewidth}{0.5mm}}
\setmainfont[Mapping=TeX-text]{CMU Serif}
\begin{document}
\begin{titlepage}
\begin{center}

\noindent\begin{minipage}[b]{.55\textwidth}
\begin{Large}
\emph{Εθνικό Μετσόβιο Πολυτεχνείο\\
Σχολή ΗΜ\&ΜΥ\\
Βάσεις Δεδομένων\\
7\textsuperscript{ο} εξάμηνο, Ροή Λ\\
Ακαδημαϊκή περίοδος: 2011-2012\\}
\end{Large}
\end{minipage}%
\begin{minipage}[b]{.45\textwidth}
     \centering
     \includegraphics[scale=0.8]{title/ntua_logo}
\end{minipage}

\vspace{5cm}
\begin{huge}
\underline{2\textsuperscript{η} Σειρά Γραπτών Ασκήσεων}
\end{huge}
\vfill

\begin{flushright}
\Large{Γερακάρης Βασίλης}\\
\large{<vgerak@gmail.com> }\\
\large{Α.Μ.: 03108092}\\
\end{flushright}
\vspace{1cm}

\large\today\\
\end{center}
\end{titlepage}


\section{Μοντέλο Οντοτήτων - Συσχετίσεων (ER model)}
Στην εκφώνηση της άσκησης μας δίνονται οι προδιαγραφές μιας βάσης δεδομένων που θα περιέχει
πληροφορίεσ για το Διεθνή Αερολιμένα Αθηνών Ελευθέριος Βενιζέλος. Με βάση την
περιγραφή αυτή μπορούμε να ορίσουμε τα παρακάτω σύνολα οντοτήτων (entities)
και συσχετίσεων (relations).

\subsection{Οντότητες (entities)}
\begin{itemize}
\item Το αεροπλάνο \emph{(plane)}. Κάθε αεροπλάνο που εξυπηρετείται από τον αερολιμένα.
Προσδιορίζεται από δύο ιδιότιτες το \emph{registration number(αριθμός
εγγραφής)} που έχει το όνομα \emph{pid} στη βάση μας και είναι το πρωτεύον
κλειδί, και τον τύπο του, \emph{tid}, που αποτελεί ξένο κλειδί.
\item Τύπος αεροπλάνου \emph{(type)}. Οι τύποι αεροπλάνων που μπορεί να εξυπηρετήσει το
αεροδρόμιο. Κάθε τύπος έχει ένα \emph{tid} ως πρωτεύον κλειδί, αριθμό επιβατών
\emph{(capacity)}, και το βάρος του \emph{(weight)}.
\item Εργαζόμενος \emph{(employee)}. Κάθε άτομο που εργάζεται στο συγκεκριμένο αεροδρόμιο. Η
οντρότητα αυτή έχει ως πρωτεύον κλειδί τον αριθμό μέλους στο εργατικό
σωματείο \emph{(umn)}. Ακόμη έχει τον αριθμό κοινωνικής ασφάλισης \emph{(ssn)},
διεύθυνση \emph{(addr)} και μισθό του εργαζομένου \emph{(salary)} εάν τα γνωρίζουμε.
\item Τύπος ελέγχου \emph{(checktype)}. Πρόκειται για κάθε δυνατό τύπο ελέγχου όπως
ορίζει η FAA. Κάθε τύπος ελέγχου έχει έναν κωδικό αριθμό \emph{(chckid)} που
αποτελεί και πρωτεύον κλειδί ένα όνομα \emph{(name)} και τη μέγιστη δυνατή
βαθμολογία \emph{(maxscore)}.
\end{itemize}
\begin{figure}[h]
\centering
\includegraphics[width=0.75\textwidth]{files/aviation_entities.png}\\
\caption{Οντότητες στο μοντέλο της βάσης.}
\end{figure}

\pagebreak
\subsection{Συσχετίσεις (relations)}
\begin{itemize}
\item Η συσχέτιση \emph{ειδικεύεται} συνδέει τον κάθε τεχνικό με τον τύπο
αεροπλάνου όπου ο τεχνικός έχει εξειδίκευση. Η συσχέτιση είναι Ν:Μ καθώς ένας
τεχνικός μπορεί να ειδικεύεται σε περισσότερα από ένα αεροσκάφη και πολλοί
τεχνικοί να έχουν ειδίκευση στον ίδιο τύπο αεροσκάφους.
\item Η συσχέτιση \emph{IS\_A} συνδέει τους τεχνικούς και τους ελεγκτές
εναέριας κυκλοφορίας με την οντότητα εργαζόμενος.
\item Η συσχέτιση ελέγχει συνδέει τους τεχνικούς με το αεροπλάνο και τον τύπο
του ελέγχου. Πρόκειτε για μια συσχέτιση Ν:Μ:Κ πολλοί τεχνικοί μπορούν να
ελέγξουν πολλά αεροσκάφη κάνοντας διαφορετικούς ελέγχους σε κάθε περίπτωση.
\end{itemize}
\begin{figure}[h]
\centering
\includegraphics[width=0.7\textwidth]{files/aviation_relations_1a.png}\\
\caption{Συσχέτηση τεχνικού και τύπου αεροπλάνου.}
\end{figure}

\begin{figure}[h]
\centering
\includegraphics[width=0.5\textwidth]{files/aviation_relations_1b.png}\\
\caption{Συσχέτηση των τεχνικών και ελεγκτών με τους εργαζόμενους.}
\end{figure}

\begin{figure}[h]
\centering
\includegraphics[width=0.9\textwidth]{files/aviation_relations_2.png}\\
\caption{Συσχέτιση μεταξύ τύπων ελέγχου, τεχνικών και αεροσκαφών.}
\end{figure}
\pagebreak

\section{Σχεσιακό Μοντέλο  (Relational model)}
\end{document}
